\documentclass[12pt,a4paper]{article}
\usepackage[pdftex]{graphicx}
\graphicspath{{Image/}}
\usepackage{epstopdf}
\usepackage{float}
\usepackage{amsmath}
\usepackage{mathtools}
\DeclarePairedDelimiter\abs{\lvert}{\rvert}
\DeclarePairedDelimiter\norm{\lVert}{\rVert}
\usepackage{esint}
\usepackage{amsfonts}
\usepackage{times}
\usepackage[top=.4 in, left=0.9in, right=0.9in]{geometry}
\usepackage{bbm}
\usepackage{dsfont}
\usepackage{enumerate}
\usepackage{titlesec}
\newcommand{\rn}{\mathbb{R}}
\newcommand{\E}{\mathbb{E}}
\newcommand{\Gn}{\mathbb{G}_{n}}
\newcommand{\G}{\mathbb{G}}
\newcommand {\tab}{\hspace{10 mm}}
\DeclareMathOperator*{\argmax}{arg\,max}
\DeclareMathOperator*{\argmin}{arg\,min}
\titleformat{\section}[block]{\large \it \filcenter}{}{0.5 ex}{}
\titleformat{\subsection}[block]{\small\it \filcenter}{}{0.5 em}{}
\titleformat{\subsubsection}[block]{\small\it \filcenter}{}{0.5 em}{}
\title{Empirical Methods in Economics\\\small{Assignment V}}
\date{Orville D. Mondal\\ $25^{th}$ of November, 2018\vspace{-3ex}}
\begin{document}
\maketitle
\begin{enumerate}[1.]
\item Values for questions 1 and 2 are -1257.0744, and -1259.5315, respectively.
\item These are the values for when $u_{i}$=0, for all $i$. Starting value is a tuple of $(\gamma,\beta,\sigma_{\beta})$
\begin{center}\begin{tabular}{c c c }
\hline \hline\vspace{2mm}
Starting Value & Argmax &Max. Value\\
\hline
(1,1,1)& (-0.5060,2.4937,1.3738) & 536.7241\\
(1,1,1) & (-0.4626,1.4363,1.8121) & 555.1852
\end{tabular}
\end{center}
\item These are the values when one allows $u_i$ to vary, while maintaining that $\mu=[\beta,0]'$, i.e. $u_i$ is mean 0. Starting value tuple is $(\gamma,\beta,\sigma_{u},\sigma_\beta,\sigma_{u\beta})$
\begin{center}\begin{tabular}{c c c }
\hline \hline\vspace{2mm}
Starting Value & Argmax &Max. Value\\
\hline
(1,1,1,1,0.5)& (-0.3923,0.9590,1.2837,1.2506,0.7741) & 530.2781
\end{tabular}
\end{center}

\end{enumerate}
\textbf{Note}: In the calculations for question 3, when using Gaussian quadrature, one uses 100 draws from a normal distribution, while when using Monte Carlo methods, one uses 500 pseudo-random draws from a normal distribution. \\For quesiton 4, integration is based on a 10,000 point grid of $(\beta,u_i)$, drawn from a $N(\mu,\Sigma)$, where $\mu=[\beta,0]$, and $\Sigma$ is a general, symmetric, positive semi-definite matrix.
\end{document}