\documentclass[12pt,a4paper]{article}
\usepackage[pdftex]{graphicx}
\graphicspath{{Image/}}
\usepackage{epstopdf}
\usepackage{float}
\usepackage{amsmath}
\usepackage{mathtools}
\DeclarePairedDelimiter\abs{\lvert}{\rvert}
\DeclarePairedDelimiter\norm{\lVert}{\rVert}
\usepackage{esint}
\usepackage{amsfonts}
\usepackage{times}
\usepackage[top=.4 in, left=0.9in, right=0.9in]{geometry}
\usepackage{bbm}
\usepackage{dsfont}
\usepackage{enumerate}
\usepackage{titlesec}
\newcommand{\rn}{\mathbb{R}}
\newcommand{\E}{\mathbb{E}}
\newcommand{\Gn}{\mathbb{G}_{n}}
\newcommand{\G}{\mathbb{G}}
\newcommand {\tab}{\hspace{10 mm}}
\DeclareMathOperator*{\argmax}{arg\,max}
\DeclareMathOperator*{\argmin}{arg\,min}
\titleformat{\section}[block]{\large \it \filcenter}{}{0.5 ex}{}
\titleformat{\subsection}[block]{\small\it \filcenter}{}{0.5 em}{}
\titleformat{\subsubsection}[block]{\small\it \filcenter}{}{0.5 em}{}
\title{Empirical Methods in Economics\\\small{Assignment IV}}
\date{Orville D. Mondal\\ $24^{th}$ of October, 2018\vspace{-3ex}}
\begin{document}
\maketitle
No estimated values of $\pi$ are mentioned here, since the final table with results for the standard errors already summarises the accuracy of the estimates. Instead, only key points of the calculations are mentioned.
\begin{enumerate}[(1)]
\item First, equi-distributed points are obtained from the \texttt{qnwequi} function in Miranda and Fackler's CE Tools collection. The calculation is based on a draw of \texttt{50,000} points in the unit square. Each point is assigned the same weight (namely, \texttt{1/50000}).
\item The quadrature points and weights used here are the most basic. Namely, there are \texttt{25,000} points on \texttt{[0,1]}, denoted $\{x_{n}\}$ with weights specified as:
\[w_{n}=
\begin{cases}
  h/3, & \mbox{if } x_{n}\in\{0,1\} \\
  4h/3, & \mbox{if } \text{n is even} \\
  2h/3, & \mbox{otherwise},
\end{cases}
\]
Where $h=\texttt{1/25000}$. The integral is calculated $iteratively$, i.e. first a value for $x_{n}$ is fixed, and the integral is calculated over the $y$ co-ordinate (using the same points and weights as above), following which the integral values are weighted together using $w_{n}$.
\item The process for this is a duplication of the one in question 1, with a minor change owing to calculating $\sqrt{1-x_n^{2}}$ at each of the equi-distributed points on \texttt{[0,1]}.
\item The same manner of duplication with minor changes, as in question 3.
\item The results of the estimation for the six methods are given below. It appears that Newton-Coates, for the Pythagorean case is the most accurate method, across the number of draws.
\begin{table}[h]
\caption{Standard Errors from $\pi$ Calculations} by number of draws\vspace{2mm}
\centering
\begin{tabular}{c c c c c}
\hline \hline\vspace{2mm}
Method &100 &1000 &10000 \\
\hline
Pseudo-MC (Indicator function)&0.0131&0.0017&1.7523e-04\\
Quasi-MC (Indicator function)&4.6624e-04&2.5365e-06&6.2830e-07\\
Newton-Coates(Indicator function)&1.2828e-04&1.1066e-07&1.5048e-10\\
Pseudo-MC (Pythagorean)&0.0065&8.2770e-04&7.7761e-05\\
Quasi-MC (Pythagorean)&1.0818e-04&8.7960e-07&1.1207e-09\\
Newton-Coates(Pythagorean)&3.7232e-05&3.7390e-08&3.7405e-11
\end{tabular}
\end{table}
\end{enumerate}
\end{document} 